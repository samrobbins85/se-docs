%% Generated by Sphinx.
\def\sphinxdocclass{report}
\documentclass[letterpaper,10pt,english]{sphinxmanual}
\ifdefined\pdfpxdimen
   \let\sphinxpxdimen\pdfpxdimen\else\newdimen\sphinxpxdimen
\fi \sphinxpxdimen=.75bp\relax

\PassOptionsToPackage{warn}{textcomp}
\usepackage[utf8]{inputenc}
\ifdefined\DeclareUnicodeCharacter
% support both utf8 and utf8x syntaxes
  \ifdefined\DeclareUnicodeCharacterAsOptional
    \def\sphinxDUC#1{\DeclareUnicodeCharacter{"#1}}
  \else
    \let\sphinxDUC\DeclareUnicodeCharacter
  \fi
  \sphinxDUC{00A0}{\nobreakspace}
  \sphinxDUC{2500}{\sphinxunichar{2500}}
  \sphinxDUC{2502}{\sphinxunichar{2502}}
  \sphinxDUC{2514}{\sphinxunichar{2514}}
  \sphinxDUC{251C}{\sphinxunichar{251C}}
  \sphinxDUC{2572}{\textbackslash}
\fi
\usepackage{cmap}
\usepackage[T1]{fontenc}
\usepackage{amsmath,amssymb,amstext}
\usepackage{babel}



\usepackage{times}
\expandafter\ifx\csname T@LGR\endcsname\relax
\else
% LGR was declared as font encoding
  \substitutefont{LGR}{\rmdefault}{cmr}
  \substitutefont{LGR}{\sfdefault}{cmss}
  \substitutefont{LGR}{\ttdefault}{cmtt}
\fi
\expandafter\ifx\csname T@X2\endcsname\relax
  \expandafter\ifx\csname T@T2A\endcsname\relax
  \else
  % T2A was declared as font encoding
    \substitutefont{T2A}{\rmdefault}{cmr}
    \substitutefont{T2A}{\sfdefault}{cmss}
    \substitutefont{T2A}{\ttdefault}{cmtt}
  \fi
\else
% X2 was declared as font encoding
  \substitutefont{X2}{\rmdefault}{cmr}
  \substitutefont{X2}{\sfdefault}{cmss}
  \substitutefont{X2}{\ttdefault}{cmtt}
\fi


\usepackage[Rejne]{fncychap}
\usepackage{sphinx}

\fvset{fontsize=\small}
\usepackage{geometry}


% Include hyperref last.
\usepackage{hyperref}
% Fix anchor placement for figures with captions.
\usepackage{hypcap}% it must be loaded after hyperref.
% Set up styles of URL: it should be placed after hyperref.
\urlstyle{same}

\usepackage{sphinxmessages}
\setcounter{tocdepth}{1}



\title{Durham Foodbank}
\date{Mar 30, 2020}
\release{1.0.0}
\author{Group 12}
\newcommand{\sphinxlogo}{\vbox{}}
\renewcommand{\releasename}{Release}
\makeindex
\begin{document}

\pagestyle{empty}
\sphinxmaketitle
\pagestyle{plain}
\sphinxtableofcontents
\pagestyle{normal}
\phantomsection\label{\detokenize{index::doc}}



\chapter{Introduction}
\label{\detokenize{index:introduction}}
To get started with this code, first clone the git repository at
\sphinxurl{https://github.com/samrobbins85/software-engineering-12} using

\begin{sphinxVerbatim}[commandchars=\\\{\}]
\PYG{n}{git} \PYG{n}{clone} \PYG{n}{https}\PYG{p}{:}\PYG{o}{/}\PYG{o}{/}\PYG{n}{github}\PYG{o}{.}\PYG{n}{com}\PYG{o}{/}\PYG{n}{samrobbins85}\PYG{o}{/}\PYG{n}{software}\PYG{o}{\PYGZhy{}}\PYG{n}{engineering}\PYG{o}{\PYGZhy{}}\PYG{l+m+mf}{12.}\PYG{n}{git}
\end{sphinxVerbatim}

Then move into the directory created

Then run

\begin{sphinxVerbatim}[commandchars=\\\{\}]
\PYG{n}{npm} \PYG{n}{install}
\end{sphinxVerbatim}

And run

\begin{sphinxVerbatim}[commandchars=\\\{\}]
\PYG{n}{npm} \PYG{n}{start}
\end{sphinxVerbatim}

to start the server


\section{System Structure}
\label{\detokenize{docs/Introduction/Backend_overview:system-structure}}\label{\detokenize{docs/Introduction/Backend_overview::doc}}
The whole system composes of three primary components: the MongoDB
Database, the Express Server (which I will also refer to as the back\sphinxhyphen{}end
server) and any number of clients running the front\sphinxhyphen{}end program.

The MongoDB database is where the data describing the contents and
structure of the warehouse will be stored so that it can be accessed and
edited by warehouse employees. We chose to use MongoDB as it is both
easily extensible and flexible. Details on the exact structure of the
database can be found in the Installation section.

The express server acts as a middle\sphinxhyphen{}man between the clients and the
MongoDB and provides a number of useful API functions to interface with
the database. It provides error handling between the client and server,
and between the server and the database, ensuring that the whole system
remains robust and to ensure consistency. Details on how to install and
deploy the express server are also in the Installation section.


\section{Back\sphinxhyphen{}end Functions}
\label{\detokenize{docs/Introduction/Backend_overview:back-end-functions}}
In this section I will give a brief overview of all back\sphinxhyphen{}end functions
in \sphinxcode{\sphinxupquote{routes/stockTake.js}}


\subsection{addTray}
\label{\detokenize{docs/Introduction/Backend_overview:addtray}}
When provided with a tray object and a MongoDB database object, this
function will add the contents of the tray object to the database.


\subsection{editTray}
\label{\detokenize{docs/Introduction/Backend_overview:edittray}}
When provided with a tray object and a MongoDB database object, this
function will update the contents of an existing tray object in the
database, provided it exists.


\subsection{removeTray}
\label{\detokenize{docs/Introduction/Backend_overview:removetray}}
When provided with the location of a tray object and a MongoDB database
object, this function will delete the tray from the database.


\subsection{switchTray}
\label{\detokenize{docs/Introduction/Backend_overview:switchtray}}
When provided with the location of two different tray objects and a
MongoDB database object, this function with swap the positions of two
trays in the database.


\subsection{addTrayMany}
\label{\detokenize{docs/Introduction/Backend_overview:addtraymany}}
When provided with an array of tray objects and a MongoDB database
object, this function will add all the tray objects to the database in a
single command. This should be used if many trays are added at once, as
adding them one\sphinxhyphen{}by\sphinxhyphen{}one will take an unacceptable amount of time.


\subsection{editTrayMany}
\label{\detokenize{docs/Introduction/Backend_overview:edittraymany}}
When provided with an array of tray objects and a MongoDB database
object, this function will edit all the tray objects that are in the
database in a single command. This should be used if many trays are
edited at once, as editing them one\sphinxhyphen{}by\sphinxhyphen{}one will take an unacceptable
amount of time.


\subsection{removeTrayMany}
\label{\detokenize{docs/Introduction/Backend_overview:removetraymany}}
When provided with an array of tray locations and a MongoDB database
object, this function will remove all the trays at the positions
specified. This should be used if many trays are deleted at once, as
deleting them one\sphinxhyphen{}by\sphinxhyphen{}one will take an unacceptable amount of time.


\subsection{getAllCategory}
\label{\detokenize{docs/Introduction/Backend_overview:getallcategory}}
When provided with a category and a MongoDB database object, this
function will return all trays with a matching category. This can be
used to quickly obtain the locations of items of a specific type.


\subsection{getNextNExpiring}
\label{\detokenize{docs/Introduction/Backend_overview:getnextnexpiring}}
When provided with a number N, an optional category argument and a
MongoDB database object, this function will return the next N expiring
trays. If a category is specified, it will return the next N expiring in
that category only. This can be useful if we wish to find the items
closest to expiring so that we can focus on taking them from the
warehouse first.


\subsection{getZones}
\label{\detokenize{docs/Introduction/Backend_overview:getzones}}
This function, when provided with a MongoDB database object, simply
returns all zones in the warehouse.


\subsection{addZone}
\label{\detokenize{docs/Introduction/Backend_overview:addzone}}
When provided with a zone object and a MongoDB database object, the zone
will be added to the database. This can be used in the initial creation
of the warehouse and to add zones temporarily when there is a need to
meet a higher than normal capacity.


\subsection{editZone}
\label{\detokenize{docs/Introduction/Backend_overview:editzone}}
When provided with a zone object and a MongoDB database object, the zone
in the database will be edited, provided that it does exist.


\subsection{removeZone}
\label{\detokenize{docs/Introduction/Backend_overview:removezone}}
When provided with a zone location and a MongoDB database object, the
zone will be removed from the database.


\subsection{addBay}
\label{\detokenize{docs/Introduction/Backend_overview:addbay}}
When provided with a bay object and a MongoDB database object, the bay
will be added to the database.


\subsection{editBay}
\label{\detokenize{docs/Introduction/Backend_overview:editbay}}
When provided with a bay object and a MongoDB database object, the bay
will be edited, provided it does exist.


\subsection{removeBay}
\label{\detokenize{docs/Introduction/Backend_overview:removebay}}
When provided with the location of the bay and a MongoDB database
object, the bay will be removed from the database.


\subsection{getTraysInBay}
\label{\detokenize{docs/Introduction/Backend_overview:gettraysinbay}}
When provided with the location of a bay and a MongoDB database object,
the function will return a list of all tray objects inside that bay,
provided the bay exists. This will be used to display bay contents in
the front\sphinxhyphen{}end application.


\subsection{getBaysInZone}
\label{\detokenize{docs/Introduction/Backend_overview:getbaysinzone}}
When provided with the location of a zone and a MongoDB database object,
the function will return a list of all the bay objects inside that zone,
provided the zone exists.


\subsection{moveTray}
\label{\detokenize{docs/Introduction/Backend_overview:movetray}}
When provided with two tray locations and a MongoDB database object, the
function will move the tray from one location to another provided a tray
exists in the start location and does not exist in the end position.


\subsection{mongoUpdate}
\label{\detokenize{docs/Introduction/Backend_overview:mongoupdate}}
This function takes a request body and a method code. It will first
connect to the MongoDB database to get the database object. Then, using
the method code, it will pass the request body to one of the functions
described above. It is then responsible for handling errors that occur
and returning the result to the front\sphinxhyphen{}end.


\chapter{Installation}
\label{\detokenize{index:installation}}

\section{Authentication}
\label{\detokenize{docs/Installation/authentication:authentication}}\label{\detokenize{docs/Installation/authentication::doc}}
Authentication for the website is handled through
\sphinxhref{https://auth0.com/}{Auth0} the free tier allows for up to 7000
users, which should be more than ample.

The process of transferring the account to one that you own is very
simple:
\begin{enumerate}
\sphinxsetlistlabels{\arabic}{enumi}{enumii}{}{.}%
\item {} 
Sign up for Auth0 \sphinxhref{https://auth0.com/signup}{here}

\item {} 
Under the applications tab on the dashboard create a new Generic
application

\item {} 
Copy the client ID from the application, as pictured below

\end{enumerate}

\noindent\sphinxincludegraphics[width=300\sphinxpxdimen]{{auth0}.png}
\begin{enumerate}
\sphinxsetlistlabels{\arabic}{enumi}{enumii}{}{.}%
\setcounter{enumi}{3}
\item {} 
Paste this value under \sphinxcode{\sphinxupquote{clientID}} in \sphinxcode{\sphinxupquote{auth\_config.json}}, which is
located in the \sphinxcode{\sphinxupquote{src}} folder of the \sphinxcode{\sphinxupquote{frontend}} server

\item {} 
Also change the domain in the \sphinxcode{\sphinxupquote{auth\_config.json}} file, this domain
can also be found in the application, as shown below

\item {} 
Disable signups by going to the connections tab of the dashboard,
clicking on \sphinxcode{\sphinxupquote{Username\sphinxhyphen{}Password\sphinxhyphen{}Authentication}}, then turning on the
disable signups option as shown below

\end{enumerate}

This is done as otherwise any user could come and sign up for the system
without permission


\section{Backend}
\label{\detokenize{docs/Installation/backEnd:backend}}\label{\detokenize{docs/Installation/backEnd::doc}}
In our handover there is a folder named \sphinxcode{\sphinxupquote{backend}}, in this is stored
the server that manages the requests made in the website. It is built
using Node.js and Express and can be started with \sphinxcode{\sphinxupquote{npm install}}
followed by \sphinxcode{\sphinxupquote{npm start}}, the server will then be hosted on port
\sphinxstylestrong{3000}

When both the front end have been started, requests can be made between
them \#\# Deployment Our recommendation for deploying this is on
\sphinxhref{https://www.heroku.com/}{heroku} as it has a good free tier and has
suitable pricing plans for expanding if usage increases.

It is also simple to set up:
\begin{enumerate}
\sphinxsetlistlabels{\arabic}{enumi}{enumii}{}{.}%
\item {} 
Create an account \sphinxhref{https://signup.heroku.com/}{here}

\item {} 
Click the new button on your dashboard to create a new application

\item {} 
Give it a name and set the region to Europe

\item {} 
Follow the on screen instructions to get your code deployed

\end{enumerate}


\section{Database}
\label{\detokenize{docs/Installation/database:database}}\label{\detokenize{docs/Installation/database::doc}}
For this application to run it also needs a database set up, for this we
have chosen \sphinxhref{https://www.mongodb.com/}{MongoDB}.


\subsection{Deployment}
\label{\detokenize{docs/Installation/database:deployment}}
Our recommendation for deployment is to use \sphinxhref{https://www.mongodb.com/cloud/atlas}{MongoDB
Atlas} this has a free tier but
allows for scaling up to a paid plan if your requirements exceed the
limits of the free tier.

To get started sign up \sphinxhref{https://www.mongodb.com/cloud/atlas}{here}
then choose any of the cloud providers, and select a local cluster with
a free tier

Once this is done, a particular structure of the database will need to
be established if you don’t want to change the code
\begin{enumerate}
\sphinxsetlistlabels{\arabic}{enumi}{enumii}{}{.}%
\item {} 
Create a database named \sphinxcode{\sphinxupquote{foodbank}}

\item {} 
Create 4 collections, \sphinxcode{\sphinxupquote{bays}}, \sphinxcode{\sphinxupquote{dummy}}, \sphinxcode{\sphinxupquote{food}} and \sphinxcode{\sphinxupquote{zones}} in
that database

\end{enumerate}

If you do want to change the names of the database or collections,
please consult the expansion section of this user guide.


\section{Frontend}
\label{\detokenize{docs/Installation/frontEnd:frontend}}\label{\detokenize{docs/Installation/frontEnd::doc}}
In our handover there is a folder named \sphinxcode{\sphinxupquote{frontend}}, in this is stored
the user interface with the program. It is a react app, and can be
started with the command \sphinxcode{\sphinxupquote{npm install}} followed by \sphinxcode{\sphinxupquote{npm start}}, you
will then be taken to a browser window with the application loaded. \#\#
Deployment The code can be deployed using any static hosting, our
recommendation is \sphinxhref{https://zeit.co/}{ZEIT Now} as they have a very
generous free tier and it is very easy to get set up. There are two main
ways of getting set up


\subsection{Git Repository}
\label{\detokenize{docs/Installation/frontEnd:git-repository}}
\sphinxstyleemphasis{If you don’t know what this is, then proceed to the other option.}

For continuous development, this is a better option as any changes you
make will be automatically deployed to the website.

This can be done in 3 steps
\begin{enumerate}
\sphinxsetlistlabels{\arabic}{enumi}{enumii}{}{.}%
\item {} 
Add the files to a GitHub, GitLab or Bitbucket repository

\item {} 
Create an account \sphinxhref{https://zeit.co/signup}{here}

\item {} 
Link together the repository and ZEIT by clicking \sphinxcode{\sphinxupquote{Import Project}}

\end{enumerate}


\subsection{Command line}
\label{\detokenize{docs/Installation/frontEnd:command-line}}\begin{enumerate}
\sphinxsetlistlabels{\arabic}{enumi}{enumii}{}{.}%
\item {} 
Install npm using the instructions
\sphinxhref{https://www.npmjs.com/get-npm}{here}

\item {} 
Create a ZEIT account \sphinxhref{https://zeit.co/signup}{here}

\item {} 
Install the Now CLI using the instructions
\sphinxhref{https://zeit.co/download}{here}

\item {} 
The code can then be deployed with the command \sphinxcode{\sphinxupquote{now}}, as documented
\sphinxhref{https://zeit.co/docs/v2/platform/deployments\#now-cli}{here}

\end{enumerate}



\renewcommand{\indexname}{Index}
\printindex
\end{document}